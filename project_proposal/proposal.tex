\documentclass{article}
% \usepackage[utf8]{inputenc}
% \usepackage[space]{xeCJK}
\usepackage{cTex}
\usepackage{amsmath}
% \usepackage{esint}
\usepackage{enumerate}
\usepackage{float}
\usepackage{booktabs}
\usepackage{geometry}
\geometry{left = 2.5cm, right = 2.5cm, top = 2.5cm, bottom = 2.5cm}
% \linespread{1.3}
\title{PROJECT PROPOSAL}
\author{XNote Developers}
\date{March 16, 2019}

\begin{document}
\maketitle
% \tableofcontents
% \newpage

\section{Project Overview}
\noindent\textbf{App Name}: XNote\\
\noindent\textbf{Team Name}: XNote Developers\\
\noindent\textbf{Intended Users}: Learners, especially students\\
\noindent\textbf{Platform}: Windows, Linux, mac OS\\
\noindent\textbf{Motivation}: In the information era, many learners choose to use note taking or mind map Apps to help their learning process. However, we found many today's learning Apps inconvenient due to their hard-to-organized structure and insufficient visualization. For example, some Apps only support text input so users cannot organize the logic structure clearly. Most Apps that support mind maps require users to draw the maps on their own, which wastes much time. And those maps are often static and cannot make association across notes. Therefore, we plan to make our own App.\\
\noindent\textbf{Description}: XNote is a cross-platform desktop application we propose to facilitate the learning process through note taking and mind maps, especially for students. Students need to take down notes quickly on a class and review them efficiently after class. For those reasons, XNote will support Markdown, a lightweight markup language with plain text formatting syntax, to relieve learners of the trouble of typesetting. It will also generate a dynamic mind map to show interconnections within and across different notes. These two features are what is lacking in the other existing note-taking Apps. 

\section{Features}
Detailed features of XNote are listed as follows:
\begin{enumerate}
	\item Support both markdown and ordinary text format.
	\item A note management system, including but not limited to key word search, time-based learning record, tagging, cloud synchronization.
	\item Generate mind maps automatically.
	\item Output html and pdf files.
	\item Personalized user interface style.
	\item Support multiple platforms, including Windows, Linux, macOS.
\end{enumerate}

\section{Technical Analysis}
We plan to use the Electron framework to develop XNote. It uses Node.js runtime for the backend and Chromium for the frontend so it enables us to use web technologies, like JavaScript, CSS and HTML to build desktop Apps. We may use the open source code from markdown official to do the markdown support part of our App. And since it is written in JavaScript so we would like to use this language for the backend of our project. For the UI, we will mainly use CSS and HTML. It will contain A text editor, a typeset view, a mind map view and a timeline to record the use. We believe a simple style is appropriate while also give the users options to create their own style.

\section{About Us}
% Table generated by Excel2LaTeX from sheet 'Sheet1'
\begin{table}[H]
  \centering
  % \caption{}
    \begin{tabular}{lllll}
    \toprule
          & \multicolumn{1}{c}{Name} & \multicolumn{1}{c}{Student ID} & \multicolumn{1}{c}{Email} & \multicolumn{1}{c}{Phone Number} \\
    \midrule
    Project manager & Jingyu Li & 517030910318 & hyperion@sjtu.edu.cn & 13917219581 \\
    \midrule
    Team member & Shixuan Gu & 517030910314 & lonzarude@sjtu.edu.cn & 18916070781 \\
          & Du Liu & 517030910346 & gavin\_liudu@163.com & 18767829001 \\
          & Yu Fan & 517030910339 & fan\_y2000@163.com & 15221280063 \\
          & Qiuxuan Ling & 517030910278 & 2017lqx@sjtu.edu.cn & 13607342714 \\
    \bottomrule
    \end{tabular}%
  % \label{tab:addlabel}%
\end{table}%

(小组成员:李竞宇、顾诗轩、刘督、范禹、凌邱璇)



\end{document}
